% --- PACOTES ESSENCIAIS ---
\usepackage[utf8]{inputenc}             % Codificação de caracteres (obsoleto no LuaLaTeX/XeLaTeX)
\usepackage[T1]{fontenc}                % Melhor suporte a hifenização e acentos
\usepackage{graphicx}                   % Inclusão de imagens
\usepackage{amsmath, amssymb, amsthm}   % Matemática avançada
\usepackage{booktabs}                   % Tabelas profissionais
\usepackage{array}                      % Controle avançado de tabelas
\usepackage{enumitem}                   % Personalização de listas
\usepackage{geometry}                   % Margens e layout da página
\usepackage{hyperref}                   % Links e metadados (SEMPRE ÚLTIMO)
\usepackage{indentfirst}
\usepackage{fancyhdr}
\usepackage{graphicx}
\usepackage[position=top]{caption}
\usepackage[labelsep=endash]{caption}
\usepackage{float}
\usepackage{color, colortbl}
\usepackage{adjustbox}
\usepackage{enumitem}
\usepackage{multicol, multirow}
\usepackage{xcolor}


% --- CONFIGURAÇÕES GLOBAIS ---
\geometry{a4paper, left=2cm, right=2cm, top=2cm, bottom=2cm}  % Margens
\definecolor{uftverde}{rgb}{0.0, 0.537, 0.486}
\captionsetup[table]{skip=2pt}   % Para tabelas
\captionsetup[figure]{skip=5pt}  % Para figuras
\usepackage{titlesec}
\titleformat{\section}
  {\normalfont\bfseries\fontsize{12pt}{14pt}\selectfont}
  {\thesection}{1em}{}

\titleformat{\subsection}
  {\normalfont\bfseries\fontsize{12pt}{14pt}\selectfont}
  {\thesubsection}{1em}{}

\titleformat{\subsubsection}
  {\normalfont\bfseries\fontsize{12pt}{14pt}\selectfont}
  {\thesubsubsection}{1em}{}


\DeclareMathOperator{\senop}{sen}
\newcommand{\sen}{\senop\,}

% --- ESTILOS DE LISTAS ---
\setlist[enumerate]{
    label=\thesection.\arabic*,
    leftmargin=0pt,
    itemindent=*,
    align=left
}

% --- ESTILOS DE TABELAS ---
\renewcommand{\arraystretch}{1.25}       % Altura padrão das linhas
\setlength{\extrarowheight}{2pt}        % Espaçamento extra em tabelas
\newcolumntype{L}[1]{>{\raggedright\arraybackslash}p{#1}} % Coluna alinhada à esquerda


% --- METADADOS (HIPERREF) ---
\hypersetup{
    colorlinks=true,
    linkcolor=blue,
    filecolor=magenta,
    urlcolor=cyan,
    pdftitle={Oscilações Harmônicas Simples},
    pdfauthor={Labotório de Ensino de Física}
}


% --- CABEÇALHOS E RODAPÉ ---
\pagestyle{fancy}

% Desativa a linha horizontal no cabeçalho
\renewcommand{\headrulewidth}{0pt}

% Ativa a linha horizontal no rodapé (ajuste a espessura)
\renewcommand{\footrulewidth}{1pt}  % ativa linha no rodapé

% Configure os conteúdos do cabeçalho e rodapé conforme desejar
\fancyhead{} % limpa cabeçalho
\fancyfoot{} % limpa rodapé
\fancyfoot[L]{Labfis - UFT - Câmpus Palmas}
\fancyfoot[R]{\thepage} % exemplo: número da página central no rodapé

\renewcommand{\footrule}{%
  {\color{uftverde}\hrule height 1pt width \headwidth}%
  \color{black} % ou cor padrão da fonte
  \vskip2pt
}

% --- REMOVE O TÍTULO AUTOMÁTICO ---
\renewcommand{\maketitle}{}